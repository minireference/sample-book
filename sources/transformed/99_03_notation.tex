%!TEX root = ../main.tex

\chapter{Notation}
\label{appendix:notation}
\vspace{-7mm}

This appendix contains a summary of the notation used in this book.


\section*{Math notation}

\begin{longtable}{rll} 
\toprule
Expression  	&	Read as  	& Used to denote			\\
\midrule
$a,b,x,y$	
	& 
	& variables \\
$=$	
	& is equal to 
	& expressions that have the same value \\ $\eqdef$
	& is defined as 
	& a new variable definition  \\
$a+b$
	& $a$ plus $b$
	& the combined lengths of $a$ and $b$ \\
$a-b$	
	& $a$ minus $b$
	& the difference in lengths between $a$ and $b$ \\
$a\times b = ab$
	& $a$ times $b$
	& the area of a rectangle   \\
$a^2= aa$
	& $a$ squared 
	& the area of a square of side length $a$ \\
$a^3= aaa$
	& $a$ cubed 
	& the volume of a cube of side length $a$ \\
$a^n$
	& $a$ to the $n$
		& $a$ multiplied by itself $n$ times 		\\
$\sqrt{a} = a^{\frac{1}{2}}$
	& square root of $a$
	& the side length of a square of area $a$ \\
$\sqrt[3]{a}= a^{\frac{1}{3}}$
	& cube root of $a$
	& the side length of a cube with volume $a$  \\
$a/b = \frac{a}{b}$
	& $a$ divided by $b$
	& $a$ parts of a whole split into $b$ parts \\
$a^{-1}= \frac{1}{a}$
	& one over $a$
	& division by $a$ 					\\
$f(x)$	
	& $f$ of $x$
	& the function $f$ applied to input $x$ 	\\
$f^{-1}$ 
	& $f$ inverse 
	& the inverse function of $f(x)$   \\
$f \circ g$ 
	& $f$ compose $g$ 
	& function composition; $f \circ g(x) = f(g(x))$   \\
$e^x$ 
	& $e$ to the $x$ 
	& the exponential function base $e$ \\
$\ln(x)$ 
	& natural log of $x$ 
	& the logarithm base $e$ 				\\
$a^x$ 
	& $a$ to the $x$ 
	& the exponential function base $a$ \\
$\log_a(x)$ 
	& log base $a$ of $x$ 
	& the logarithm base $a$ 				\\
$\theta,\phi$
	& \emph{theta}, \emph{phi}
	& angles 					\\
$\sin,\cos,\tan$
	& sin, cos, tan 
	& trigonometric ratios 			\\
$\%$
	& percent
	& proportions of a total; $a\%=\frac{a}{100}$ 		\\
\bottomrule
\end{longtable}

